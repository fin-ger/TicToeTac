
\documentclass{article}
\usepackage[utf8]{inputenc}
\usepackage[T1]{fontenc}
\usepackage[ngerman]{babel}
\usepackage{geometry}
\pagenumbering{Roman}
\usepackage{fancyhdr}
\usepackage{lastpage}
\usepackage{graphicx}
\usepackage{amsmath}
\usepackage{amssymb}
\usepackage{listings}

\pagestyle{fancy}
\fancyhf{}
\lfoot{Johann Wagner, Marten Wallwein-Eising,\\ Johannes Wünsche, Paul Stang}
\cfoot{Page \thepage \hspace{1pt} of \pageref{LastPage}}
\geometry{
	a4paper,
	total={170mm,257mm},
	left=20mm,
	top=20mm,
}

\lstset{ 
	tabsize=3,
	language=c
}

%\newcommand{\supervec}[3]{\ensuremath{\begin{pmatrix} #1 \\ #2 \\ #3 \end{pmatrix}}}
%\newcommand{\normalvec}[2]{\ensuremath{\begin{pmatrix} #1 \\ #2 \end{pmatrix}}}

\begin{document}

\title{TicToeTac - Learning TicTacToe with the LMS algorithm}
\author{Johann Wagner, 212276 \\ Johannes Wünsche, 211720 \\ Marten Wallewein-Eising, 212277 \\ Paul Stang, insert matr.Nr here...}
\date{\today}
\maketitle

\section{Overview}
Play against random Player, not Smart Player...

\section{Move decision}
In each turn our player goes through all possible moves he can make and calculate the board score for these to get the best move he can choose. 

\section{Learning}
Our player learns each time a match is finished comparing the calculated board score to defined scores for win (100), draw(50) and lose(-100). We thought about learning in each step comparing the current board to the last board, but we can not be sure if the calculation of our last board was close to the target function in the first matches.

\section{LMS Parameters}
Our function which will be approximated is defined as $f: Board->Score$, so we calculate a score of the current board state depending on the current calculated weights $w_0,..,w_5$. We decided to use the following parameters for our LMS function.
\begin{itemize}
	\item $x_1$: Count of own rows, which can be completed
	\item $x_2$: Count of enemy rows, which can be completed
	\item $x_3$: Count of min marks required to fill a row
	\item $x_4$: Count of min enemy marks required to fill a row	
\end{itemize} 
These parameters are extracted from the board for every possible move that can be made to estimate the best move regarding the current board state. We did some calculation with example boards to set the weights to useful initial values to avoid starting on complete wrong weights. The learning rate was therefore choosen very small with $\mu = 0.001$.


\end{document}